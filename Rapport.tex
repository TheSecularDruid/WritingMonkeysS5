\documentclass{article}

\usepackage[french]{babel}
\usepackage[utf8]{inputenc}

\title{Shakespearian Monkeys}

\begin{document}

\titlepage

\tableofcontents

\section{Introduction}

\emph{Le singe n’a autant l'air d'un animal que lorsqu'on l'affuble de vêtements d'homme.\\
-Proverbe indien ; Le livre des proverbes de l'hindoustani (1988)}\\

Nous avions à réaliser en langage C un programme informatique permettant de simuler des singes, travaillant sans relâche à créer du texte en anglais.
Pour cela, les primates s'aideraient d'un texte source dans lequel était déjà écrit des mots en anglais, d'un artiste peu connu mais prometteur : Shakespeare. (L'idée étant qu'ils "s'inspirerait" de son travail afin de créer leur propre texte). Dans cette entreprise, les singes ont chacun un travail bien précis :
\begin{enumerate}
    \item Le singes lecteur, lisant le fichier source et l'enregistrant dans une structure de donnée.
    \item Le singes statisticien, dont le travail consiste à analyser les mots enregistrer et à en sortir certaines données statistiques.
    \item Le singes écrivain, qui a certainement le travail le plus critique : s'inspirer des mots lus pour créer un texte neuf.
    \item Le singe imprimeur, qui doit imprimer le travail de l'écrivain.
\end{enumerate}

Nous avions à réaliser un programme "par palier", comprenez ici qu'au début, le programme ne doit respecter qu'un cahier des charges réduits, qui augmente en difficulté au fur et à mesure des "Achievements". Notre équipe n'est allé qu'a l'Achievement 2. 


\newpage

\section{Structure du Makefile}

\section{Conception et développement de la base}

\section{Conception et développement de l'Achievement 1}

\section{Conception et développement de l'Achievement 2}

\section{Conclusion}

\end{document}
